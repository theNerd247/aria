\documentclass[main.tex]{subfiles}
\lstset{breakatwhitespace=false,
	breaklines=true,
	captionpos=top,
	numbers=left,
	frame=tl
	}

\begin{document}

\begin{appendices}

\section{Robot Source Code}
\label{ap:appendA}
Below are the source code files for the various robot programs. Each program is
implemented as its own function which is called from the main function of the
program. Below is the main function. The main function is called with the
\code{RUNASNx} macro. This macro is replaced with the desired program function
call during compile time (this is done through a makefile system which is beyond
the support of this report\footnote{\url{https://www.gnu.org/software/make/}}). 

\lstinputlisting[language=c++,title=main.cpp]{/home/noah/class/mech/aria/src/main.cpp}

\subsection*{Square Robot Code}
\lstinputlisting[language=c++,title=SquareBot.cpp]{/home/noah/class/mech/aria/src/asn1/asn1.cpp}
	
\subsection*{Sonar Measure Robot}
\lstinputlisting[language=c++,title=SonarBot.cpp]{/home/noah/class/mech/aria/src/asn2/asn2.cpp}

\subsection*{S-Curve Robot}
\lstinputlisting[language=c++,title=SCurveBot.cpp]{/home/noah/class/mech/aria/src/asn4/asn4.cpp}

\end{appendices}

\end{document}
