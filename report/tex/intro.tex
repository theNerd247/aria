\documentclass[main.tex]{subfiles}
\begin{document}
\section{Introduction}

\subsection{Dead Reckoning}
% explain the issues of dead Reckoning
% use as an introduction: 
% 	to the FWKin prob.
% 	to the different sensors used

%todo: fix 1st sent.

In order for a robot to move effectively about an environment it must have
information about its current motion state (position, velocity, heading, etc.).
For example for a two wheeled robot that attempts to move from one position in a
room to another needs to keep track of its current heading, position, speed, and
acceleration. 

Dead reckoning is a solution to this problem that we will explore. Dead
reckoning is the determining of the location of an object based on a known
motion model and current state information. Once the current state of an object
is known then its motion model can be used to predict future motion states. 

A two wheeled robot with a caster (or free wheel used for balance) uses the
following odometry model:

\begin{equation}\label{eq:deadReckonState}
	\begin{bmatrix}
	\frac{1}{2}	& \frac{1}{2} \\[0.3em]
	\frac{1}{\mathrm{w}} & \frac{1}{\mathrm{w}}
	\end{bmatrix} 
	\begin{bmatrix}
	s_R \\ s_L
	\end{bmatrix}
	=
	\begin{bmatrix}
	S \\ \theta
	\end{bmatrix}
\end{equation}

Where $S$ and $\theta$ respectively are the robot distance travelled and heading
of the robot. $s_L$ and $s_R$ are the distances traveled by each wheel (left and
right respectively).

While~\eqref{deadReckonState} is useful it is difficult to use in determining
the motion state of a robot. To simplify our analysis we will assume the
following: 

\begin{itemize}
\item The robot's motion model is considered on a time differential \delt that
is small with respect to the total time the robot performs its motion. 

\item The robot's heading $\theta$ is constant on time differential \delt while
translating. That is the robot always travels in a straight line.
\end{itemize}

From these assumptions we have a motion model for the robot:

\begin{equation}
\label{eq:motionState}
\xi = \xi_{0} +
\begin{bmatrix}
S\cos{\theta} \\
S\sin{\theta} \\
\theta
\end{bmatrix}
\end{equation}

\subsection{Forward Kinematics}
% use diff. physics models. 
% matrix form? (think of A from kalman filter)
% affects of error in movement 
% maybe mention the kalman filter?

\subsection{Wheel Encoders}
% usages
% simple design of quad Encoders
% sensor readings format (grayscale encoding) 
% measuring x/v/a from encoders
% sources of error

\subsection{Sonar Range Finder}
% usages
% simple design explanation (signal threshold and time of flight)
% error types 

\end{document}
