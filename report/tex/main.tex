\documentclass[a4paper,titlepage]{article}
\usepackage{labreport}

\begin{document}

	\title{Mobile Robots Lab Report}
	\author{Noah Harvey}
	% Date is automagically included (see Latex documentation)
	\maketitle

	\begin{abstract}

	The robot localization problem is a key problem faced in modern day mobile
	robots. This problem involves obtaining ``clean'' data about the robot's
	environment, determining the current location of the robot, and deciding on an
	optimal travel path towards a desired location. One method of solving the
	robot localization problem is the dead reckoning method. This method uses
	received information from its environment via sensors and then uses the sensor
	data and previous motion state information to determine its current location. 

	Mobile robot simulation software is a useful tool for designing and testing
	localization algorithms. Aria is a C++ SDK used for developing and testing
	mobile robot software. MobileSim is a simulation software for testing mobile
	robot software which uses Aria. Both Aria and MobileSim were used to develop
	and test mobile robot software. This report aims to explore the method of dead
	reckoning using the Aria mobile robots software, wheel encoders and sonar
	range finders. 
	
	\end{abstract}

	\tableofcontents
	\listoftables
	\listoffigures

	\pagebreak
	% other sections of the labreport go here
	\subfile{intro}
	\subfile{methods}
	\pagebreak
	\subfile{appendixA}
\end{document}
